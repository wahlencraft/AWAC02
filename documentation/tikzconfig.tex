\usepackage{tikz}
\usepackage{xparse}
\usetikzlibrary{positioning, fit, backgrounds, shapes, arrows.meta}

% State machine

\tikzstyle{base} = [
    minimum width=3cm,
    minimum height=1cm,
    node distance=2em,
    text centered,
    draw=black,
    rounded corners=0.8em,
    inner sep=5pt]

\tikzstyle{start} = [base, rectangle, rounded corners, draw=black, fill=red!30]

\tikzstyle{io} = [trapezium, trapezium left angle=70, trapezium right angle=110,
 minimum height=1cm, text centered, draw=black, fill=blue!30, trapezium stretches=true,
 align=center, inner sep=1em]

\tikzstyle{state} = [base, rectangle, fill=orange!30]

\tikzstyle{decision} = [base, rectangle, rounded corners=0.8em, fill=green!30]

\tikzstyle{subprocess} = [rectangle, draw = black!50, minimum height = 2em, fill=pink!50]

\tikzstyle{process2} = [rectangle, minimum height=1cm, align=center, draw=black, fill=yellow!30, inner sep=1.8em]

\tikzstyle{process_label} = [external/.style={draw=black!50}, font={\fontsize{13pt}{12}\selectfont}]

\tikzstyle{arrow} = [thick,->,>=stealth]
\tikzstyle{slow_arrow} = [dotted, thick,->,>=stealth]
\tikzstyle{line} = [thick]

\NewDocumentCommand\StateWithBullets{m m m >{\SplitList{;}}m}{%
    \node (#1) [state, text width=6cm, #2, inner sep=5pt] {\large #3\small \begin{itemize}
    \ProcessList{#4}{\insertitem} \end{itemize}};
}
\newcommand\insertitem[1]{\item #1}

\NewDocumentCommand\JustBullets{m m >{\SplitList{;}}m}{%
    \node (#1) [state, text width=3.5cm, #2, inner sep=5pt, rounded corners,
    fill=blue!30] {\vspace{-0.8em}\small\begin{itemize}
    \ProcessList{#3}{\insertitem} \end{itemize}};
}

% Distances
\newcommand{\shortarrowlength}{3.1em}

% Menu
\tikzstyle{menu-base} = [text centered, draw=black, inner sep=0.5em, node
distance=4em, minimum height=3em, minimum width=9em]

\tikzstyle{menu-enterexit} = [menu-base, rectangle, rounded corners, fill=black!10]

\tikzstyle{menu-item} = [menu-base, rectangle, rounded corners, minimum height=3.5em]

\tikzstyle{menu-small-item} = [menu-base, rectangle, rounded corners, minimum
width=4em, minimum height=2em]

\tikzstyle{menu-setting} = [menu-base, rectangle, rounded corners, minimum
width=7em]

\tikzstyle{button-label} = [circle, draw=black, minimum size=1.3em, fill=white, inner
sep=0, line width=0.5pt, font=\footnotesize]

\tikzstyle{menu-single} = [->, >={Latex[length=2mm, width=3mm]}, line width=4pt]
\tikzstyle{menu-double} = [menu-single, <->]

\newcommand{\LRvert}[2]{%
    \draw[menu-single] ([xshift=1em]#1.south) -- node[button-label] {R}
        ([xshift=1em]#2.north);
    \draw[menu-single] ([xshift=-1em]#2.north) -- node[button-label] {L}
        ([xshift=-1em]#1.south);
}

\newcommand{\LRhor}[2]{%
    \draw[menu-single] ([yshift=0.5em]#1.east) -- node[button-label] {R}
        ([yshift=0.5em]#2.west);
    \draw[menu-single] ([yshift=-0.5em]#2.west) -- node[button-label] {L}
        ([yshift=-0.5em]#1.east);
}

\newcommand{\IncrementDecrementNode}[2]{%
    \node (_#2anchor) [#1=\shortarrowlength+1.3em of #2, outer sep=1pt, inner sep=0] {};
    \node (_#2L) [button-label, above left=0.2em of _#2anchor] {L};
    \node (_#2dec) [below left=0.2em of _#2anchor, inner sep=0, circle, minimum size=1.3em] {--};
    \node (_#2R) [button-label, above right=0.2em of _#2anchor] {R};
    \node (_#2inc) [below right=0.2em of _#2anchor, inner sep=0, circle, minimum size=1.3em] {+};
    \node (_#2set) [fit={(_#2L) (_#2R) (_#2dec) (_#2inc)}, draw] {};
}

\newcommand{\SetTimeNode}[3]{%
    \node (#1) [#2, menu-small-item] {#3};
    \IncrementDecrementNode{below}{#1}
    \draw[menu-double] (#1) -- node[button-label] {SP} (_#1set.north);
}

